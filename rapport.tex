\documentclass[12pt]{article}
\usepackage [utf8]{inputenc}
\usepackage [T1]{fontenc}
\usepackage [francais]{babel}

\author{LABBE Emeric, NERESTAN Clément, Pellegrini Charles}
\title{Rapport projet Programmation Système}

\begin{document}
\maketitle
\tableofcontents

\newpage
\section{Commandes internes}

\subsection{echo}
Renvoie la chaine de caractère qui vient d'etre entrée.

\subsection{date}

\subsection{cd}

\subsection{pwd}
Récupère le répertoire actuel avec la commande get\_current\_dir\_name() et l'affiche.

\subsection{history}

\subsection{kill}

\subsection{exit}
Quitte le shell en transmettant le code de retour (ici atoi(c[0]))

\newpage
\section{Remote Shell}
\subsection{Gestion d'une liste de machines (add, remove, list)}

\subsection{Exécution sur un unique shell faussement distant}

\subsection{Exécution sur plusieurs shell distants avec ssh}

\subsection{Affichage dans des fenetres séparées avec xcat.sh}

\newpage
\section{Expressions}
On utilise deux fonctions, evaluer\_expr et exec\_expr.
La fonction evaluer va faire un fork et va appeller exec. La fonction exec va évaluer le type de l'expression et va adapter son comportement à celle ci.

\subsection{expression ; expression}
Séquences d'instructions.
\\
e->type = SEQUENCE

Chaque instruction séparée par ; est traitée.

\subsection{expression || expression}
Evaluation de la première expression, si celle ci retourne 0 (est considérée comme vraie), la suivante ne s'exécute pas.
\\
e->type = SEQUENCE\_OU

\subsection{expression \& \& expression}
Les deux expressions doivent etre vraies.
\\
e->type = SEQUENCE\_ET

\subsection{(expression)}
Création d'un sous shell dans lequel va s'exécuter l'expression.
\\
e->type = SOUS\_SHELL

\subsection{expression \&}
Tache en arrière plan.
\\
e->type = BG

\subsection{expression | expression}
Pipe des expressions.
\\
e->type = PIPE

\subsection{expression > fichier}
Redirection de la sortie vers le fichier spécifié.
\\
e->type = REDIRECTION\_O

\subsection{expression < fichier}
Redirection de l'entrée du fichier vers l'entrée standard.
\\   
e->type = REDIRECTION\_I

\subsection{expression >> fichier}
Redirection de la sortie en mode APPEND.   
\\ 
e->type = REDIRECTION\_A   

\subsection{expressions récursives}
Toutes les expressions sont récursives.

\newpage
\section{Bilan}

\newpage
\section{Apport du projet}
Ce projet nous a permis d'utiliser tout ce qu'on nous avons appris cette année en programmation système, et de voir concretement comment cela se passe dans un shell.
Nous avons approfondi nos connaissances sur la gestion des processus, notament l'utilisation du fork et la gestion des redirections des entrées et sorties.
Nous avons pu aussi revoir les différents signaux ainsi que leur utilisation en fonction du besoin.

\end{document}
