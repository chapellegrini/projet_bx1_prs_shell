\documentclass[12pt]{article}
\usepackage [utf8]{inputenc}
\usepackage [T1]{fontenc}
\usepackage [francais]{babel}

\author{LABBE Emeric, NERESTAN Clément, Pellegrini Charles}
\title{Rapport projet Programmation Système}

\begin{document}
\maketitle
\tableofcontents

\newpage
\section{Commandes internes}

\subsection{echo}
Renvoie la chaine de caractère qui vient d'etre entrée.

\subsection{date}

\subsection{cd}

\subsection{pwd}
Récupère le répertoire actuel avec la commande get\_current\_dir\_name() et l'affiche.

\subsection{history}

\subsection{kill}

\subsection{exit}
Quitte le shell en transmettant le code de retour (ici atoi(c[0]))

\newpage
\section{Remote Shell}
\subsection{Gestion d'une liste de machines (add, remove, list)}

\subsection{Exécution sur un unique shell faussement distant}

\subsection{Exécution sur plusieurs shell distants avec ssh}

\subsection{Affichage dans des fenetres séparées avec xcat.sh}

\newpage
\section{Expressions}
On utilise deux fonctions, evaluer\_expr et exec\_expr.
La fonction evaluer\_expr regarde si l'expression est une commande interne : dans ce cas elle appèle directement
la fonction correspondante. Autrement elle se fork, et la ligne 
\begin{verbatim}
exit(exec_expr(e));
\end{verbatim}
fait que le fils se quittera forcement après l'appel à exec\_expr.
\newline Le père, lui, appèle directement waitpid avec le pid du fils.
ainsi, il ne peut pas y avoir de processus zombie : le fils est récupéré par le père,
et les processus éventuellement crées par le fils seront récupérés par init.\newline
Cette implementation a un autre avantage : on ne redirige les descripteurs de fichiers que des processus fils,
il n'y a pas besoin de les sauvegarder avant de les modifier.


\subsection{instruction simple}
e->type == SIMPLE\\
L'expression est à executer directement.
Le processus ayant déjà été forké, il suffit d'appeler execvp.

\subsection{expression ; expression}
e->type == SEQUENCE\\
L'expression de gauche est passée en paramètre à evaluer\_expr, celle de droite à exec\_expr.
Dans le cas ou celle de gauche est une commande interne, elle peut modifier le comportement de celle de droite
(par exemple cd peut changer le répertoire courant de l'expression de droite)

\subsection{expression || expression}
e->type == SEQUENCE\_OU\\
Évaluation de la première expression, si celle-ci retourne 0 (est considérée comme vraie),
la suivante ne s'exécute pas. Encore une fois, l'expression de gauche est passée en paramètre à evaluer\_expr,
celle de droite à exec\_expr.

\subsection{expression \& \& expression}
e->type == SEQUENCE\_ET\\
Comme pour SEQUENCE\_OU, mais on évalue la première expression,
si elle est fausse on retourne l'évaluation de la deuxième, sinon on ne l'évalue pas mais on retourne vrai.

\subsection{(expression)}
e->type == SOUS\_SHELL\\
Création d'un sous shell dans lequel va s'exécuter l'expression. Ici, on appèle simplement evaluer\_expr.

\subsection{expression \&}
e->type == BG\\
Tache en arrière plan. On fork et le fils appèle exec\_expr, le père n'attend pas le résultant.
Le père se terminant de suite après, le fils est récupéré par init.

\subsection{expression | expression}
e->type == PIPE\\
Pipe des expressions. Le père et le fils appèlent tous les deux exec\_expr et non evaluer\_expr,
les forker une fois de plus serait inutile.

\subsection{expression > fichier}
e->type == REDIRECTION\_O\\
Redirection de la sortie vers le fichier spécifié.

\subsection{expression < fichier}
e->type == REDIRECTION\_I\\
Redirection du contenu du fichier vers l'entrée standard.
   

\subsection{expression >> fichier}
e->type == REDIRECTION\_A\\
Redirection de la sortie en mode APPEND.   


\newpage
\section{Bilan}

\newpage
\section{Apport du projet}
Ce projet nous a permis d'utiliser tout ce qu'on nous avons appris cette année en programmation système, et de voir concretement comment cela se passe dans un shell.
Nous avons approfondi nos connaissances sur la gestion des processus, notament l'utilisation du fork et la gestion des redirections des entrées et sorties.
Nous avons pu aussi revoir les différents signaux ainsi que leur utilisation en fonction du besoin.

\end{document}
